\documentclass[11pt]{article}

\begin{document}
\title{COLLEGE OF COMPUTING AND INFORMATION SCIENCES}
\maketitle

\begin{center}\begin{large}GROUP MEMBERSHIP \end{large} \end{center}

\begin{tabular} {|c|c|c|c|}
\hline
SNo & Names & Registration  Number & Signature \\ \hline
1 & ABULE AUGUSTINE ARUMADRI &  15/U/2633/PS &  \\ \hline
2 & AYIKO Jeremiah Sara &  15/U/186 & \\ \hline
3 & OWOMUGISHA ISAAC &  15/U/12351/PS & \\ \hline
3 & SENDIKADDIWA MARVIN &  15/U/1154 & \\ \hline
\end{tabular}

\tableofcontents

\section{Introduction}
	\subsection{Background}
	\subsection{Problem Statement}
	\subsection{Objectives}
		\subsubsection{Main Objective}
		\subsubsection{Specific Objective}
	\subsection{scope}
		Computer programs and source code exist in many programming languages and size (in
		terms of actual lines of code), with different computer resource needs to efficiently execute.
		In order to serve a broad variety, it is important to build a general platform that can
		accommodate as many languages as possible. The aim of this project is to build a proof of
		concept platform. Thus, we only consider the C and Java programming languages.
		However, in the future we want the system to be platform independent. \\ \\
			When designing a platform used for automatic grading purposes, accurate assessment is of
		utmost importance. Given this, automatically generated programming assignments must be
		tailored with the intention of keeping the auto grading process simple and stable, at the
		same time challenging and relevant to students in order to hone their programming skills
		at the same time pass that particular course unit. Another important aspect is user feedback.
		The system is designed to be a substitute of the lecturer when it comes to assessment.
		Therefore it must provide relevant feedback to the students regarding use of better or
		correct algorithms, and also to the lecturer on topics where students are finding challenges.\\ \\
			For purposes of testing the effectiveness of the working system, we will focus on
		programming course units taught in the Computer Science course only within Makerere
		University. Programming languages covered in this course are C and Java.

	\subsection{Significance}
			A delivered working system will be of great help to both students and lecturers who use it
		as part of their teaching tools. Many at times students do not get the personal attention
		from lecturers as they so do desire. This may be attributed to the large number of students
		offering a particular course unit, each most likely with their unique challenges they face.
		There also exists the general perception that university students are not supposed to be
		spoon-fed, but rather engage in rigorous research and extensive personal reading. Much
		as this encourages academic independence, many students need one on one academic
		guidance. Our auto grading system provides this guidance. \\ \\
			By having the ability to suggest better algorithms and language structures/functions to use,
		each student can be guided on better coding techniques and how to optimize their programs. In the current setup at 				CIT, this is very difficult to achieve through manual means by the lecturer. \\ \\
			By enabling the lecturer to give in-class assignments that can be automatically graded,
		students are encouraged to constantly practice programming skills in order to be prepared
		for these assignments. This has a direct positive impact on their academic performance in
		that particular course unit, at the same time improving their general programming skills
		even though not examinable. 


\section{Literature Review}
	In this section, we provide a summary the previous work that has been done on E-learning platforms, auto-graders and online 			judges.\\ \\
		Many of the leading universities of computer science education around the world incorporate auto grading in their 			qprogramming and algorithms courses. For example, in 2013, MIT researchers Singh et al [11] introduced a new method for 			automatically providing feedback for introductory programming problems. According to [12] due to fair and timely feedback 			results from online 	judge websites, online practice outperforms traditional programming practice.

	\subsection{Related work}
		As far as E-learning is concerned in Ugandan Universities, the used platforms are not that efficient in reducing 				instructors efforts i.e. the Makerere University E-learning Environment (MUELE) platform is used for only publishing and 				collecting assignments for students. An extra effort has to be done on manually grading the submitted assignments.\\ \\
		The MUELE platform is owned by Makerere University and strictly bound to Makerere University Students.  Due to the 			Unfavorable factors constraining the platform, it will be impossible to extend the project to the platform.\\
		Designing and developing our project from scratch has problems overweighing advantages. One of these is rendering 			the project futile due to the competitions of other dominant E-learning platforms. Another problem associated with this is the 			development time required, a lot of time will be needed to get the design, user accounts, privacy and other security issues 			right if we are to create a project from scratch. Thus, this leaves us on cooperating and improving the available open source 			projects/platforms.\\ \\
		One of the most dominant E-learning platform used worldwide, Moodle. “Moodle is a free and open-source software 			learning management system written in PHP and distributed under the GNU General Public License. Developed on pedagogical 			principles, Moodle is used for blended learning, distance education, flipped classroom and other e-learning projects in schools, 			universities, workplaces and other sectors.” ("Moodle", 2017). The Moodle platform is composed of several features such 			“components” that enable its dynamism and thus can be adopted for the project. \\ \\
		“Plugins are a flexible tool set, allowing Moodle users to extend the features of the site.” ("Moodle", 2017). Plugins for 			moodle can be thought of modules or widgets with specific functionality. Currently moodle is maintained by  thousands of 			plugins developed to suite different teaching curriculum. Amongst the plugin developed sofar include:- URKUND plagiarism 			plugin for checking plagiarism in submitted files, reengagement for sending timely feedbacks to students reminding them of 			uncompleted course activities. Some of these plugins being open source just as the platform they run (moodle), they can be 			used as sub modules in our plugin. For example, in order to solve the plagiarism detection stage in our project, the plagiarism 			plugin can be used as a sub component. \\ \\
		The most popular application of online judges is in programming contests such as TopCoder,
	ACM/ICPC and Google Code Jam. The most prominent of these is the ACM/ICPC [1]. The
	ACM/ICPC consists of algorithmic and programming problems for contestants to solve. The
	students’ submissions are evaluated based on the output of their programs. A typical problem in
	this contest consists of the following parts:
		\begin{itemize}
			\item Problem description: This is a description of the problem to be solved.
			\item  Input description: This gives the specification of the input file from which the user’s
				solution programs read data
			\item Output description: This specifies the format in which the program should produce
				results.
			\item Sample input and sample output: These provide examples to clarify the input and output
				specifications.
		\end{itemize}

		For each problem in ACM/ICPC, there are multiple sets of data to be processed. This is an
	important feature that serves two major purposes [12]: first, they are used to test that the program
	can work in all possible situations (including corner cases). Second, they are used to calculate the
	running time of a program (a good solution should be able to work on both small datasets and
	larger datasets). The use of multiple datasets therefore ensures that a student’s submission is
	evaluated comprehensively. \\
	Test datasets of problems in the ACM/ICPC are either generated by hand for small-sized datasets
	or by programs written by jury members that generate datasets according to some predefined
	patterns or at random [3].\\
	In [7], Dr. Antti Laaksonen describes how a competitive programming approach using an online
	judge to teach algorithm concepts has helped improve the problem solving skills of the students at the University of Helsinki. In 		the course, the TMC system [9] is used. TMC allows students to
	create their Java solutions in the NetBeans IDE and evaluates the solutions using JUnit tests. The
	tests are written by the course staff. In this paper, he describes how this model of teaching helps
	students understand the importance of developing efficient algorithms. He also points out some
	limitations of using online judges. He gives an example of the following problem from [5]:
	Describe an $O(n)$-time algorithm that, given a set $S$ of $𝑛$ distinct numbers and a positive
	integer $k ≤ n$, determines the $k$ numbers in $S$ that are closest to the median of $S$.
	He points out the difficulty of automatically determining whether a student’s submission of the
	above problem is really $O(n)$, since the same problem can be solved in $O(n log n)$ time. This
	therefore means that auto-graders are not perfect, however they seem to be better than traditional
	means of evaluation as the following section shows.

	\subsection{Benefits of auto-grading and online practice oriented teaching of programming}
		Researchers Wang GP et al. conducted two sets of experiments [12] in order to evaluate the
	effectiveness and validity of using their online judge system called OJPOT as compared to
	traditional teaching methods. They set out to answer the following questions [12]:
		\begin{itemize}
			\item  Does OJPOT work effectively to enhance students’ practical abilities compared to the
					traditional teaching idea?
			\item  In which aspects can OJPOT improve the student’s practical abilities?
			\item Output description: This specifies the format in which the program should produce
				results.
		\end{itemize}
		In the first set of experiments, two classes were chosen: one class defined as the control class (CC)
	and another defined as the experimental class (EC). In the CC, the traditional teaching idea was
	applied; while in the EC, the OJPOT online judge was used [12]. Before the experiment, both
	classes were given a pre-test to evaluate their elementary knowledge in C programming. Then
	during the semester, both classes were taught by the same teacher with the same timetable.
	The following data were collected during the experiment: pre-test scores, online practice and
	statistical data, post-test scores and course project and scores. Below is a summary of the results
	of their experiment ([12]):
		\begin{itemize}
			\item Both classes had similar programming foundation before the experimen
			\item At the end of the semester, there were significant differences between the CC class and
				the EC class: the EC class has overall better results in the final tests and project.
		\end{itemize}
		They conclude their results by remarking that the EC class performed better because the online
	judge system made the students more enthusiastic about the material they were studying. It was
	also observed that the general programming abilities of the EC class were improved much better
	than those of the control class


\section{Methodology}
	\subsection{Auto-grader/on-line judge}
		\subsubsection{Method of grading}
		\subsubsection{Limiting resources}
		\subsubsection{Tools available for the lecturer}
	\subsection{Analytics and other evaluation metrics}
	\subsection{Moodle}

\section{References}

\section{Appendices}
	\subsection{Appendix A: Activities Gantt chart.}
	\subsection{APPENDIX B: Financial Requirements}
\end{document}